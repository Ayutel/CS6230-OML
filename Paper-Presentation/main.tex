\documentclass{beamer}
\usepackage{amsmath}
\usepackage{amssymb}
%\documentclass{article}
%\usepackage{beamerarticle}
\usetheme{CambridgeUS}

\title[]{Newton-Type Methods for Non-Convex Optimization Under Inexact Hessian Information}
\subtitle{Presented as a part of\\
CS6230 - Optimization Methods in Machine Learning}

\author{Vishwak Srinivasan \and Ayushi Patel}

% Delete this, if you do not want the table of contents to pop up at
% the beginning of each subsection:
\AtBeginSubsection[]
{
  \begin{frame}<beamer>{Outline}
    \tableofcontents[currentsection]
  \end{frame}
}

% Let's get started
\begin{document}

\begin{frame}
  \titlepage
\end{frame}

\begin{frame}{Outline}
  \tableofcontents
  % You might wish to add the option [pausesections]
\end{frame}

% Section and subsections will appear in the presentation overview
% and table of contents.
\section{Introduction}
\subsection{The problem statement}

\begin{frame}{The problem statement}
  \begin{itemize}
  \item {
    Consider a unconstrained optimization problem
    \begin{equation}
        \min_{\mathbf{x} \in \mathbb{R}^{d}} F(\mathbf{x})
    \end{equation}
    where \(F : \mathbb{R}^{d} \rightarrow \mathbb{R}\), is \textit{smooth} and \textit{non-convex}.
  }

  \item<2-> {
    Many known methods to help solve the \emph{convex} version.
    \begin{itemize}
        \item First order methods
        \item Second order methods
    \end{itemize}
  }

  \item<3-> {
    First order methods are ``fine" - advances in tensor computing, automatic differentiation and so on.
    \begin{equation}
        \mathbf{x} := \mathbf{x} - \eta \nabla F(\mathbf{x})
    \end{equation}
    \pause
  }
  \item<4-> {
    Are second order methods just as fine?
    \begin{equation}
        \mathbf{x} := \mathbf{x} - \eta (\nabla^{2} F(\mathbf{x}))^{-1} \nabla F(\mathbf{x})
    \end{equation}
    \uncover<5->{How do we compute \(\nabla^{2} F(\mathbf{x})\)?}
  }
  \end{itemize}
\end{frame}

\subsection{Issues with the problem statement and motivation}

\begin{frame}{Issues with the problem statement}
  \begin{itemize}
  \item {
    Existing bounds for first order methods are for \emph{convex} \(F\).
    \begin{itemize}
      \uncover<2->{\item Are these methods even applicable to a \emph{non-convex} setting, as defined above?}
      \uncover<3->{\item Are there any mathematical guarantees?}
    \end{itemize}
  }

  \item<4-> {
    Second order methods: computation of \(\nabla^{2} F(\mathbf{x})\)
    \begin{itemize}
      \uncover<5->{\item How hard is this? Time Complexity Analysis}
      \uncover<6->{\item How hard is this? Space Complexity Analysis}
    \end{itemize}
  }
  \end{itemize}

  \uncover<7->{
    In Summary:\\
    Gradient based methods might not be able to optimize non-convex problems efficiently, exact hessian based methods could take a long time and a lot of space.
  }
\end{frame}

\section{Second Main Section}

\subsection{Another Subsection}

\begin{frame}{Blocks}
\begin{block}{Block Title}
You can also highlight sections of your presentation in a block, with it's own title
\end{block}
\begin{theorem}
There are separate environments for theorems, examples, definitions and proofs.
\end{theorem}
\begin{example}
Here is an example of an example block.
\end{example}
\end{frame}

% Placing a * after \section means it will not show in the
% outline or table of contents.
\section*{Summary}

\begin{frame}{Summary}
  \begin{itemize}
  \item
    The \alert{first main message} of your talk in one or two lines.
  \item
    The \alert{second main message} of your talk in one or two lines.
  \item
    Perhaps a \alert{third message}, but not more than that.
  \end{itemize}
  
  \begin{itemize}
  \item
    Outlook
    \begin{itemize}
    \item
      Something you haven't solved.
    \item
      Something else you haven't solved.
    \end{itemize}
  \end{itemize}
\end{frame}



% All of the following is optional and typically not needed. 
\appendix
\section<presentation>*{\appendixname}
\subsection<presentation>*{For Further Reading}

\begin{frame}[allowframebreaks]
  \frametitle<presentation>{For Further Reading}
    
  \begin{thebibliography}{10}
    
  \beamertemplatebookbibitems
  % Start with overview books.

  \bibitem{Author1990}
    A.~Author.
    \newblock {\em Handbook of Everything}.
    \newblock Some Press, 1990.
 
    
  \beamertemplatearticlebibitems
  % Followed by interesting articles. Keep the list short. 

  \bibitem{Someone2000}
    S.~Someone.
    \newblock On this and that.
    \newblock {\em Journal of This and That}, 2(1):50--100,
    2000.
  \end{thebibliography}
\end{frame}

\end{document}


