\documentclass{article}
\usepackage[margin=0.73in]{geometry}
\usepackage{amsmath}
\usepackage{amssymb}
\usepackage{graphicx}
\usepackage{float}
\usepackage{hyperref}

\title{Report for Programming Assignment 2}
\author{Vishwak Srinivasan\\
\texttt{CS15BTECH11043}}
\date{}

\begin{document}
\maketitle

\section*{Part b}
\subsection*{Question 1}
\begin{flushleft}
Below is the table representing the objective value of the optimal solution. The program used \texttt{CVXPY}, a convex programming module in Python.
\begin{center}
\begin{tabular}{|c|c|}
\hline
Penalty Parameter combination \((C_{1}, C_{2})\) & Objective Value \\
\hline
\hline
\((1, 1)\) & \(9.60436\)\\
\hline
\((1, 10)\) & \(30.26522\)\\
\hline
\((10, 1)\) & \(11.89106\)\\
\hline
\end{tabular}
\end{center}
\end{flushleft}

\subsection*{Question 2}
The reason of different decision boundaries occuring with variation in the penalty parameters could be attributed to the relative weights assigned to the classes. For example, if we set \(C_{1} = C_{2} = 1\), we are implicitly stating that both the classes \(\{+1, -1\}\) are equally important. On the other hand, if we assign \(C_{1} = 10, C_{2} = 1\), then we are implicitly stating that \(+1\) gets a higher priority over \(-1\), in the sense that we can't afford to mis-classify many of \(+1\) datapoints, but we could rather compromise with \(-1\) datapoints. An analogous case follows for \(C_{1} = 1, C_{2} = 10\), wherein we ``weigh'' \(-1\) more than \(+1\). The variation can be seen as the decision boundary tries to keep a higher number of examples to the class having a higher penalty on one side of itself.

\end{document}
